\documentclass{article}

\title{Linera Algebra 1}
\date{2022-18-10}
\author{Jan Buehne, Marcelo Pedernera}
\usepackage{gensymb}
\usepackage{amsmath}
\usepackage{enumitem}
\usepackage{hyperref}
\usepackage{graphicx}
\newlist{todolist}{itemize}{2}
\setlist[todolist]{label=[\hspace{0.2cm}]}

\newcommand{\theSpace}{3}

\begin{document}

    \maketitle
    \pagenumbering{gobble}
    \newpage
    \pagenumbering{arabic}

    \tableofcontents

    \section{Aussagenlogik und Mengenlehre}
    
    \subsection{Aussagenlogik}
    Eine Aussage nimmt einen eindeutigen Wahrheitswert an, entweder wahr (w) oder falsch (f). \\

    \begin{tabular}{ |c|c|c|c|c|c|c| }
        \hline
        A & B & $\lnot$A & A $\land$ B & A $\lor$ B & A $\Rightarrow$ B & A $\Leftrightarrow$ B \\
        \hline
        w & w & f & w & w & w & w \\
        w & f & f & f & w & f & f \\
        f & w & w & f & w & w & f \\
        f & f & w & f & f & w & w \\
        \hline


    \end{tabular} \\

    Ein Prädikat ist eine Aussage \( A = A(x) \), deren Wahrheitswert von einer Variablen $x$ abhängt.
    Dann sind $\forall x : A(x) (für alle x gilt A(x)) und \in x : A(x)$ (es existiert ein x, sodass A(x) gilt) Aussagen.

\end{document}